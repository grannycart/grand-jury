\documentclass[letterpaper]{article}
\pagestyle{plain}
\pagenumbering{gobble} % eats all page numbers
% \usepackage[left=4.5cm,top=2.5cm,right=4.5cm,nohead,nofoot]{geometry} 
\renewcommand{\baselinestretch}{1.0}
% \usepackage[colorlinks=true,linkcolor=red]{hyperref} % Hyperlink capabilities
\renewcommand{\rmdefault}{cmss}  % Changes fonts to sans
\widowpenalty=10000
\clubpenalty=10000 % against widows and orphans.

\usepackage{hyperref}




\begin{document}

\textbf{Mark Torrey, Long-form writing sample:}
\textit{I wrote this after serving on a grand jury in Brooklyn in 2015. I wanted to both document what I had learned from the experience so I could share it with others who have to serve on a grand jury --- which I am happy to say I have done so a number of times now. This is unrelated to my work at CUP.}\\
\\
% \title{A Progressive Guide to Grand Jury Duty in NYC}
% The negative vspace is to move the title up, and keep latex from putting the last paragraph on its own page
% \author{Mark Torrey\\ (dynohub@gmail.com)}
% \date{February, 2016}
% \maketitle

%%%%%%%%%%%%%%%%%%%%%%%%%%%%%%%%%%%%%%%%%%%%%%%%%%%%%%%%%%%%%%%%%%%%
% \large
%%

% \renewcommand{\abstractname}{Executive Summary} % Renames abstract section
%\begin{abstract}
% Abstract goes here, if it's useful.
%\end{abstract}


So you got picked for a grand jury.
Don't fret. It's not a disaster.
Grand jury duty was one of the most tense, stressful, rewarding, and empowering civic experiences of my life.
It's a rare opportunity for ordinary citizens to exercise real power to change the world for the better, particularly if you are inclined towards progressive politics.
But to avoid two weeks of slogging through lawyers reading legal documents at you just to vote unanimously and unquestioningly with your other jurors, you need to know \emph{how} to apply your power as a grand juror.
And trust me, nobody at the courthouse is going to fill you in on this.

What the heck is a grand jury anyway? 
What it's \emph{not} is a trial.
There won't be lawyers from opposing sides dueling it out, trying to win you over with with charismatic showmanship while dropping shocking revelations about the other side.. 
To get to a trial like that, first the government needs to file formal charges against a person accused of a crime. 
And that is done by convincing a grand jury there is enough evidence to charge the person with a crime. 
The grand jury bit is required by the Constitution.
The idea is that accused people aren't charged with a crime by the \emph{government}, they are charged by the \emph{people}, the people in this case being you: the grand jury.
What the Constitution doesn't require is for accused people or their lawyers to be present.
The person accused of a crime has no chance to defend herself to a grand jury (except in the rare cases where they can choose to testify --- but they still don't get a lawyer to help them).
The only people presenting a case to the grand jury is the government's lawyers, (usually Assistant District Attorneys --- ADAs) who present only evidence in favor of charging the accused.
The formal charge against the accused is called an indictment.
All of this stuff will be explained to you at the courthouse, but it's important to know from the start what legal environment you are heading into.

What won't be explained to you is that the government fully expects you to indict, preferably quickly..
The vast majority of grand jury cases are indicted.
You need to know that you have a small but very real power over the government in a grand jury.
You can use this power to make the government work harder for its indictments, and in doing so you can add a little more real justice to our system.
This isn't to say a little cynicism isn't appropriate.
Toward the end of my jury service, a young fellow juror said, "The thing I learned from grand jury duty is that the law can really be bent to mean anything." 
Like many other parts of the legal system, the grand jury process includes a vast difference between a person's rights under the law, and the day-to-day conventions of how people are processed by the legal system.
This goes for both you as a juror and for the person accused of a crime.
Knowing your rights and opportunities as a grand juror can help keep the system focussed on the rights of the accused rather than the conventions of prosecution.


\section*{It's you versus the Assistant District Attorneys}
Some concepts will be explained to you. 
Some concepts will be explained many MANY times because it's required by the law.
Like the fact that grand juries are not trial juries. 
You will be reminded over and over again that you are not deciding innocence or guilt, you are only deciding whether to charge someone with a crime. 
They will tell you the guilt or innocence of the accused will be determined by a trial jury, not you.
The formal process, and your responsibilities as a juror will be made mind-numbingly clear. Other concepts will be glossed over, or left out entirely.
You are being managed like cattle.
They deal with thousands of jurors every week.
They have lots of practice getting you to do what they want: hand out indictments.
To make the grand jury process efficient, the professionals who run it actively \emph{avoid} tipping you off to the larger context of the system you are participating in. 

This is how the process works: the ADA presents a small slice of the evidence against the accused to you, the grand jury. 
Most of the time this doesn't take very long, a case might be presented to the grand jury in a single 20 minute presentation.
The ADA is the only person presenting this to you, there is no judge in the room.
Since the ADA's job is to get an indictment they only show you evidence that is convincing. 
(And legal --- if a judge reviews the case later and determines the ADA presented something illegally, the whole case can be thrown out.)
The grand jury is then left alone to decide if the ADA presented enough evidence to bring charges against the accused.
If you decide to indict the person on charges, the person has the \emph{right} to take the case to a trial jury. 

But that person is almost certainly going to plead guilty to some or all of the crimes you charged him with, in exchange for less punishment. 
90-95\% of cases are settled by plea bargaining.\footnote{See: \url{https://www.bja.gov/Publications/PleaBargainingResearchSummary.pdf}}
That means after the grand jury charges the accused with a crime, the accused agrees to admit she committed the crime in exchange for a lighter sentence.
They ``bargain'' with the government's lawyers, who have the upper hand because you gave them a formal charge.
The vast majority of cases do \emph{not} end with the big dramatic trial that is the cornerstone of courtroom drama television.
The grand jury hearing is the \emph{only} trial most people will ever get, and they aren't even in the room. 

Because there is \emph{nobody} arguing for the accused, your responsibility as a grand jury is direct opposition to the ADA.
You are the skeptical observer for the people keeping the government's power in check. 
The ADA's job is to convince you that there is a good reason to charge the defendant. 
It's you versus the ADA. 
Your mission as someone interested in justice is to make the ADA's job really hard.

\section*{The ADA is not your friend}
Some ADAs are funny and charismatic. 
Some are hard calculating professionals. 
Some are attractive and have a smile that melts your tender heart. 
And some are more or less incompetent. 
None of these people are your friend. 
Don't fall for the trap of trying to please the ADA. 
You might be spending a lot of time watching this ADA.
They seem friendly and trustworthy. 
The ADAs do their best to make it appear that they know all the evidence, rules, and procedures, and your job is simply to give them their indictment. 
You will hear them repeatedly say, ``as your legal counselor'' as if the purpose is to remind you they are the only person in the room with a law degree. 
This does not mean they have your best interest in mind, or that fairness is their goal. 
This just means they can legally only tell you certain things. 

You will likely start to feel like you want to make this professional, good-looking, charismatic lawyer happy, so he will get drinks with you later or something. 
Don't fall for that. 
Their job is to get indictments out of the grand jury. 
Your job is to only give them indictments they have earned by proving that the indictment is fair and just.

\section*{The ADA \emph{will} lie to you}
Over and over again the ADA will say, ``remember it is not my recollection that controls, but your recollection.  Having said that, \emph{my} recollection is that\ldots''
This is his immunity card, after saying that phrase, he is legally allowed to tell you whatever he wants --- because it's just his ``recollection'', get it?
He will tell you whatever it is he \emph{wanted} you to hear from the evidence presented, whether it was actually included or not. 

It is easy to catch them at this, they will do it on nearly every single case, particularly if you ask them many questions about the evidence.
This is why you should ask them tons of questions before they leave you for deliberations.
It's your chance both to review the evidence, and get them to tell you what story they were \emph{hoping} the evidence would suggest.
In many cases, the story they want you to hear won't line up with the evidence as actually presented. 
If you manage to trip them up like this, often the truth will rise to the surface, and make your deliberations much easier.
When you suspect the ADA lied to you, don't doubt yourself, \emph{your} recollection probably is correct, as the immunity phrase says.
During deliberations ask your other jurors what they remember, if most of you agree, then what you collectively remember probably is the truth, not what the ADA told you. 

\section*{Take tons of notes}
On your first day you will see a video in which a judge says, ``taking notes is neither encouraged nor forbidden'' and you might think ``great! I'm off the hook for notetaking!''
Ignore that. Take as many notes as you can. 
Write down \emph{everything} because, as explained above, you can't trust the ADA. 
To make things even more confusing, most of the evidence is going to be presented to you in 15-20 minute segments which is mixed with possibly dozens of cases you have heard over your time on the grand jury. 
(I'm not going to say they do this on purpose --- that would be just a little too conspiratorial.)
It is very difficult to keep track of everything, so develop a good note-taking system that will let you put all your notes together for a given case quickly.
It can be tough to do because, on top of the cases being presented to you in short increments over many days, you are required to use the court's little blue essay books for notes.
(They shred the books at the end of your jury duty.)

You'll get advance warning about which case you are about to hear evidence for when the ADA walks into the room.
There will be a different ADA for each case most of the time, so you can start to associate a specific case with an ADA, and when that ADA comes in, start reviewing your notes.
If you can't remember the details of a case, ask your fellow jurors during the inevitable paper-shuffling minutes before the lawyer starts talking.
But the ADA isn't going to want to let you talk to your other jurors at this point, so you have to sneak it in quickly.
The ADA will review what you've heard already if you ask them too, but remember your fellow jurors will be more accurate than what the ADA tells you.

Most evidence is going to be testimony from witnesses or police officers.  
Write down absolutely everything they say. 
Some evidence, like lab reports, will be printed on paper. 
You don't have to write down everything read off of papers, but make sure to ask the ADA to leave those papers with you for your deliberations. 
Make sure you ask the ADA to \emph{leave} the papers.
Otherwise they might just hold them up for you to see.
You have the right to have the evidence in front of you during deliberations, but many ADAs don't want to give you the opportunity to review the evidence too closely if they can avoid it.

During your orientation they will tell you that you can get the court reporter to read back parts of the testimony, which is another thing that might lead you to think you can slack off on taking notes.
It sounds cool, right? Just like courtroom dramas on television.
While this is technically possible, they normally won't do it unless you take a vote in your deliberations and 12 jurors would like to hear a readback. 
This almost never happens, which means you will never get to hear the testimony a second time.
The evidence before you is only your notes, your recollection, and the occasional lab report.
(By the way: the courtroom reporters are super friendly people with interesting stories to tell.
It's well worth the effort to make friends with them during down-time.)


\section*{Don't feel like you have to do what the law says}
You will be instructed over and over that you have to apply the law without sympathy and whether you agree with the law or not. 
This is simply not true.
Grand juries have \emph{unqualified} power to decline to indict.\footnote{Check out this paper in the Cornell Law Review, it had a tremendous influence over my approach to grand jury duty, and I consider myself extremely lucky to have read it beforehand: \url{http://www.lawschool.cornell.edu/research/cornell-law-review/upload/FairfaxGrandJuryDiscretion.pdf}} 
That is why grand jury deliberations are entirely secret. 
It is a moral question, up to you, whether you \emph{want} to apply the law as it stands, or not. 
There are good reasons not to indict even if the law and the evidence is clear. Does the accused remind you of your own drug dealer, and you just don't feel it is right for them to be charged with a crime for have a pound of marijuana in their trunk? Dismiss! 
Don't let instructions from the ADA make you think you have to indict because of the law.
The charges and instructions to the grand jury are very \emph{very} carefully worded to make it seem like you have to indict if the law because the law was broken.
But if you pay close attention, you will notice that the instructions are worded to tell you that you \emph{do not} have to do what the law says.
They will tell you that you ``may'' indict someone if there is substantial legal evidence and reasonable cause to believe they did it, but you ``must'' dismiss them if those two things are not true.  
Just remember that you \emph{always} have the right to not indict. 

Think of it this way: you are doing them a favor if you find to dismiss --- it probably means their evidence was weak, the law was a really stupid law, or the ADA presents himself as an asshole that juries don't like. 
If any of those things are true, and the case went to trial, it is likely the trial jury would dismiss the case anyway. 
So you are saving that ADAs the trouble of potentially going to a long expensive trial that they would just lose. Your discretion will make for better cases.
Or you might be saving people guilty of relatively minor offenses from being forced to plea bargain.


\section*{Ask questions of the ADA}
Constantly ask the ADA questions. 
Remember, they are trying to obfuscate any facts that might make you doubt your decision to indict.  
It's you versus the ADA. 
Ask all the questions you can possibly think of. 
If they give you an answer you don't understand, ask another question. 
When it comes time to deliberate, the ADA will first give you a list of definitions of concepts that are relevant to the charges they are presenting. 
These are often counterintuitive things like the fact that an ``armed handgun'' includes weapons that are empty, but were found sitting next to ammo that could be used with them.

Sometimes they include definitions that \emph{aren't} really relevant to the charges, or maybe they are relevant to one charge but the ADA tries to make it seem like they apply to all charges. 
\emph{All} of the definitions and charges are read to you in twisted legalese.
Ask the ADA to read the definitions repeatedly until you understand what they are saying, and make sure you know which charges the definitions apply to.  
The charges are the most complicated part. 
First you will get the name of the charge, and then the requirements the evidence has to meet to charge the person with that crime. 
The hard part is that there might be a long list of charges and the difference between charges might be just a word or two. 
Like Burglary in the 2nd degree might be defined as ``stole something off a person'', while Burglary in the 3rd degree is just ``stole something''. 
And there's some (frankly insane) law that prevents them from giving you a written copy of the charges. 
So ask the ADA to read them. 
And then ask the ADA to read them again. 
Ask the ADA to explain the difference between charges that are similar (some ADAs will try to explain them in plain language, others will just read the legalese version again). 
Just keep asking until everyone on the jury understands.

Up until the moment when they give you the charges, you don't know which evidence you heard was relevant to what charge they are going to give. 
Now that you know the charge, ask the ADA to clarify evidence you head that you need to make a decision. 
But don't forget that the ADA \emph{will} lie to you and tell you only the things they ``recollect''.  
Consult your notes and see if you think they ADA is being accurate at this point.

\section*{Form a coalition}
There are 23 people on a grand jury. 
At least 12 jurors have to vote for indictment to file the charges. 
That means a lot of people can vote to dismiss the charges (or not even be there) and charges could still be filed.
It isn't like Twelve Angry Men where you need a unanimous vote so one person with an opposing view has the power to hold everything up.
On a grand jury, most of the time even if you vote for dismissing charges, you'll just be in the minority and charges will be filed anyway.

Because of this, you want to form a coalition of like-minded jurors.
This makes the down-time between cases (of which there will probably be a lot) very important.
It's your chance to talk to your fellow jurors and find out what their interests are in general, and more importantly what their values might be as jurors.
Even if you find only a few people share your values, standing together during deliberations will give you a much better chance of swaying twelve or more people to your position than if it's just you standing alone.

My jury had a large minority that was interested in reducing the number of people who end up involved with the criminal justice system for relatively minor offenses.
We managed to repeatedly convince the larger portion of the jury to dismiss charges for small amounts of drugs and for criminal charges that stemmed from simply having friends engaged in illegal activities and being in the wrong place at the wrong time.
Not on every case --- 23 people randomly selected in NYC will inevitably yield a diverse range of political views --- but enough times that I came out of jury duty with a strong sense of having made a difference in the world.


\section*{Come to a decision}
Do your best, as a jury, to reach a decision on every charge presented to you, whether for or against.
This is important because if you reach a decision the ADA no longer has any power over you.
But if you can't reach a decision, the ADA has the power to badger you, forcefully reading the instructions to try to make you feel like you aren't doing your civic duty, or sending you back into deliberations. 
There's a limit to how much they can do this because a judge will review the record later and could decide the ADA was badgering the jury too much, but it is not fun to have a professional lawyer using all of their persuasive powers to try to bully you into reaching a decision. 
If you can reach a decision, indict or dismiss, the ADA can't say anything further to you. 
If the ADA disagrees with your decision (which would be a dismissal since their job is to get indictments) all they can do is shake their heads and walk away.

\section*{Ask for more evidence}
If you seriously can't come to a decision, the ADA will likely act annoyed.
In addition to badgering you, they will eventually ask if more evidence might help you reach a decision.
Unless the problem is a moral split on your jury, it's a good idea to say yes.
On my jury there were multiple cases where we could not reach a decision because of the weak evidence presented.
The ADAs came back a few days later with stronger evidence, and we were able to reach a decision to indict.
These are usually cases where the ADAs had simply been too lazy to gather all the evidence the first time.
One of your jobs is to hold ADAs to a very high professional standard.

% Need some kind of wrap up here.



%%%%%%%%%%%%%%%%%%%%%%%%%%%%%%%%%%%%%%%%%%%%%%%%%%%%%%%%%%%%%%%%%%%%
 
\end{document}
